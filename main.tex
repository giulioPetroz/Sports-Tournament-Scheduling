\documentclass{article}
\usepackage{graphicx} % Required for inserting images
\usepackage{amsmath}
\usepackage{booktabs}
\usepackage{amssymb}
\usepackage{hyperref}
\usepackage{float} % Table placement
\usepackage{bbm}

\title{Combinatorial Decision Making And Optimization}
\author{
  Cirone Cono, \texttt{cono.cirone@studio.unibo.it}
  \and
  Dardini Jacopo, \texttt{jacopo.dardini@studio.unibo.it}
  \and
  Formichella Gio, \texttt{gio.formichella@studio.unibo.it}
  \and
  Petrozziello Giulio, \texttt{giulio.petrozziello@studio.unibo.it} 
}
\date{}

\begin{document}

\maketitle


\section{Introduction}
% Add book reference
% Add complexity reduction analysis
% \cite{10.1007/10704567_6}
In this project, we address the sports scheduling (STS) problem by applying a unified modeling approach inspired by \cite{10.1007/10704567_6} across four computational paradigms: Constraint Programming (CP), Boolean Satisfiability (SAT), Satisfiability Modulo Theory (SMT) and Mixed Integer Programming (MIP).
We begin by considering the structure of a round robin tournament: for $N$ teams, the tournament spans $N-1$ weeks, with each week consisting of $\frac{N}{2}$ games (or periods), ensuring that every team plays exactly once per week. This structure naturally corresponds to a 1-factorization of the complete graph $K_N$ on $N$ vertices \cite{Dinitz2006RoundRobin}.

A balanced tournament design is a permutation of the order of games within each week so as to satisfy the additional constraints \cite{Lamken2006BalancedTournamentDesigns}. 
Therefore it is natural to model the problem by considering an $\frac{N}{2}\times(N-1)$ matrix of matches $(t_i, t_j)$ representing the round robin tournament for $N$ teams, where each column represents a week and contains the set of matches scheduled for that round.
The objective is to find an indexing of the games along the periods of each week to satisfy the STS problem.
Our experimental results show that the time required to precompute the underlying round-robin schedule is negligible (less than a second) with respect to the one needed for all the solved instances.
Moreover, by precomputing the round robin tournament matrix, we eliminate symmetries across weeks and fix slot assignments, which significantly reduces the search space for both satisfiability and optimization.
% All implementations take as input the number of teams in the tournament and start by precomputing the weekly matchups for the round-robin tournament. 
% The precomputation cost is minimal, for all tested team sizes the precomputation required less than a second. The matchup matrix not only speeds up the search by reducing the problem to ordering matchups, but also guarantees that the constraints of each team playing each other and playing once a week are satisfied. With all four techniques, we aim to find a solution satisfying the constraints and minimizing the objective variable D, the maximum difference between games played at home and away for a team; $D\in [1, n-1]$, where n is the number of teams.

\subsection{Notation}
\begin{itemize}
  \item $N$: number of teams
  \item $T=\{t\ |\ t \in [1, N]\}$: team identifiers
  \item $P=\{p\ |\ p \in [0, ..., \frac{N}{2} - 1]\}$: period identifiers
  \item $M=\{m\ |\ m \in [0, ..., \frac{N}{2} - 1]\}$: weekly matchup identifiers
  \item $W=\{w\ |\ w \in [0, ..., N-2]\}$: week identifiers
  \item $S=\{s\ |\ s \in [0,1]\}$: slot identifiers, where $s=0$ corresponds to playing at home and $s=1$ to playing away
  \item $rb_{m, w, s}=t$: team $t$ plays in week $w$ in match $m$ in slot $s$.
  % \item $m=rb_{p, w}$ is the match $(t_1, t_2)$ in period $p$ of week $w$, where $t_1$ plays at home and $t_2$ plays away
  
\end{itemize}

\section{CP Model}
\subsection{Decision Variables}
This model utilizes two primary decision variable matrices to construct the tournament schedule and assign home/away teams for each match.

\subsubsection{\text{matches}}
For each period $p \in P$ and week $w \in W$
$$
matches_{p,w}=m \in M
$$
determines that $rb_{m, w}$ is scheduled in week $w$ and period $p$.


\subsubsection{\text{flip\_slot}}  
For each period $p \in P$ and week $w \in W$  
$$  
flip\_slot_{p,w} = a \in \{0, 1\}  
$$  
determines which team is assigned as the home team for the match $(t_1, t_2)$ scheduled in week $w$ and period $p$. Specifically, $a = 0$ means that $t_1$ plays at home while $t_2$ away; on the contrary, $a = 1$ means $t_2$ plays at home and $t_1$ away.


\subsection{Auxiliary Variables}
\subsubsection{\text{home\_games}}  
For each team $t \in T$  
$$  
home\_games_{t} = h \in [0, \dots, N-1]  
$$  
determines the total number of home games assigned to team $t$ throughout the tournament. Its value is derived from the assignments of the decision variables $matches$ and $flip\_slot$.


\subsection{Objective Function}

The model's objective is to minimize  the $\text{max\_imbalance} \in [1, \dots, N-1]$ variable quantifying the maximum absolute disparity between home and away game assignments across all teams. 
The lower bound of 1 is the best possible balance score, since $\text{max\_imbalance}=0$ is not achievable for an odd number of games, $N-1$.
The upper bound of $N-1$ represents the theoretical maximum possible deviation, occurring if a team plays all its games either at home or away.

The value of $\text{max\_imbalance}$ is determined by looking for the minimum integer satisfying the following fairness constraints:
\[ \forall t \in T : \left| 2 \times \text{home\_games}_{t} - (N-1) \right| \leq \text{max\_imbalance} \]
This formulation is derived from the difference between a team's home games and away games: $|\text{home\_games}_t - \text{away\_games}_t|$. Since \\ $\text{home\_games}_{t} + \text{away\_games}_{t} = N-1$ (total games), expressing $\text{away\_games}_t$ as a function of the other quantities and substituting it in the starting expression, yields $2 \times \text{home\_games}_{t} - (N-1)$. 
By enforcing that this score must be less than or equal to $\text{max\_imbalance}$ for every team, we constrain the variable to being grater or equal than the greatest individual team imbalance.

The matchups inside $rb$ have an unbalanced structure: the team with the largest identifier always plays at home, causing all the other teams to also have an imbalanced schedule. To guide the search towards more balanced solutions, we introduce the following constraint:
\[
\forall w \in W : \left| \sum_{p \in P} \text{flip\_slot}_{p, w} - \left\lfloor \frac{|P|}{2} \right\rfloor \right| \leq 1.
\]
This condition balances, within each week, the flip\_slot values, guiding the solver towards assigning home games more equitably, ultimately speeding up the search for an optimal solution.

\subsection{Constraints}
\subsubsection{Core Constraints}
These constraints are strictly necessary for defining a feasible round-robin schedule:
\begin{enumerate}
    \item \textbf{Each match is assigned to a unique period each week:} 
    \[ \forall w \in W : \texttt{all\_different}([\texttt{matches}[p, w] \mid p \in P]) \]
    Ensures that for every week, every match of $rb$ is scheduled to no more than one period and every period is assigned to a single match. 

    \item \textbf{Each team plays at most twice in the same period:} 
\[ \forall p \in P, \forall t \in T : \left| \{ (w, s) \mid w \in W, s \in S, \text{rb}_{\text{matches}_{p, w}, w, s} = t \} \right| \leq 2 \]
Implemented using the \texttt{global\_cardinality} global constraint.

\end{enumerate}

\subsubsection{Channeling Constraints}
\begin{enumerate}
 \item \textbf{Counting home\_games for each team:}
\[ \forall t \in T : \text{home\_games}_{t} = \sum_{p \in P, w \in W, s \in S} \mathbb{I}\left( \text{rb}_{\text{matches}_{p, w}, w, s} = t \land \text{flip\_slot}_{p, w} = s \right) \]
The indicator function $\mathbb{I}(\cdot)$ ensures that 1 is added to the sum if team $t$ is located in slot $s$ and that slot $s$ is designated as the home slot by $\text{flip\_slot}_{p,w}$.
\end{enumerate}


\subsubsection{Implied Constraints}
\begin{enumerate}
   \item \textbf{Each team appears exactly once per week:}  While this is implicitly ensured by the combination of the \texttt{all\_different} constraint on \texttt{matches} and the structure of \texttt{rb}, this constraint helps the solver propagate information earlier and prune the search space more effectively.  It is implemented using the \texttt{global\_cardinality} global constraint:
    \[
     \forall w \in W, \forall t \in T : \left| \{ (p, s) \mid p \in P, s \in S, \text{rb}_{\text{matches}_{p, w}, w, s} = t \} \right| = 1.
 \]
\end{enumerate}



\subsubsection{Symmetry Breaking Constraints}

\begin{enumerate}
\item \textbf{Period symmetry:}
Swapping period labels across weeks yields equivalent schedules, i.e rescheduling games already scheduled for period i to period j and those for period j to i results in another valid assignement. To eliminate this redundancy, we enforce a lexicographic ordering:
$$
(\text{matches}_{p, w})_{p \in P, w \in W} \succ_{\text{lex}} (\text{matches}_{p, w})_{\text{reversed}(p) \in P, w \in W}
$$
% Due to the \texttt{all\_different} constraint on $\text{matches}_{p,w}$ per week, this comparison typically resolves early by comparing $\text{matches}_{0,0}$ and $\text{matches}_{|P| - 1,0}$. However, we prefer the full lexicographic constraint over a simple pairwise comparison because it provides stronger constraint propagation improving solver performance overall.

\item \textbf{Fix first match home:} 
Swapping home/away assignments of all matches yields an equivalent optimal solution. To counteract this symmetry, we enforce the following assignment:
\[ \text{flip\_slot}_{0, 0} = 0 \]


\end{enumerate}

\subsection{Validation}

\subsubsection{Experimental Design}

The model was implemented in MiniZinc and validated through a series of experiments designed to assess solver performance under various model configurations and search strategies.

\textbf{Solvers:}
The following solvers were employed: \textit{Gecode 6.3.0}, \textit{Chuffed 0.13.2} and \textit{OR-Tools CP-SAT 9.12.4544}. 
The same time limit of $300$ seconds was imposed for each individual problem instance.

\textbf{Model Configurations:}
Four configurations were tested: \texttt{baseline} (core constraints only), \texttt{baseline+implied}, \texttt{baseline+symmetry breaking}, \texttt{full model}.

\textbf{Search Strategies:}
Three distinct search strategies were employed to analyze solver behavior

\begin{enumerate}
    \item \textbf{Default Search Strategy (Solver's Default):} Each solver relied entirely on its built-in decision heuristics and restart policies, serving as a baseline for their inherent capabilities.

    \item \textbf{Sequential Custom Search Strategy:} A manually defined sequential search \texttt{seq\_search} with \texttt{restart\_luby(100)} policy was applied. 
    Firstly the \texttt{matches} variables  are assigned and secondly the \texttt{flip\_slot} variables. 
    We chose as variable assignment strategies \texttt{dom\_w\_deg} for \texttt{matches} and \texttt{first\_fail} for \texttt{flip\_slot}, while for value assignments we used \\\texttt{indomain\_min} for both.

    \item \textbf{Relax-and-Reconstruct (LNS) Strategy:} This higher-level strategy incorporates \texttt{relax\_and\_reconstruct} on the \texttt{matches} variables (preserving 60\% of solution values), leveraging the Large Neighborhood Search (LNS) technique. It was layered on top of the "Sequential Custom Search Strategy".
\end{enumerate}

\textbf{Solver-Specific Strategy Application:}
To ensure a fair and controlled comparison under single-threaded conditions (aligning with project constraints), OR-Tools CP-SAT was run without multi-threading. For both Chuffed and OR-Tools CP-SAT, the \texttt{free\_search} parameter was explicitly omitted when applying the Sequential Custom Search and Relax-and-Reconstruct strategies. This allowed direct evaluation of the user-defined MiniZinc search annotations, rather than the solvers' embedded heuristics.


\subsubsection{Experimental Results}
In the following tables, we report the objective values found by each solver for the following four model configurations:
\begin{itemize}
    \item \texttt{bs} denotes the baseline model
     \item  \texttt{noIMPL} denotes the baseline with symmetry breaking constraints
    \item \texttt{noSB} denotes the baseline with implied constraints
    \item \texttt{complete} denotes the full model with both implied and symmetry breaking constraints
\end{itemize}

\begin{table}[htbp]
\centering
\small
\resizebox{\textwidth}{!}{%
\begin{tabular}{c|cccc|cccc|cccc}
\toprule
\textbf{n} & \multicolumn{4}{c|}{\textbf{GECODE}} & \multicolumn{4}{c|}{\textbf{CHUFFED}} & \multicolumn{4}{c}{\textbf{CP-SAT}} \\
\cmidrule(lr){2-5}\cmidrule(lr){6-9}\cmidrule(lr){10-13}
  & bs & complete & noIMPL & noSB & bs & complete & noIMPL & noSB & bs & complete & noIMPL & noSB \\
\midrule
6 & \textbf{1} & \textbf{1} & \textbf{1} & \textbf{1} & \textbf{1} & \textbf{1} & \textbf{1} & \textbf{1} & \textbf{1} & \textbf{1} & \textbf{1} & \textbf{1} \\
8 & \textbf{1} & \textbf{1} & \textbf{1} & \textbf{1} & \textbf{1} & \textbf{1} & \textbf{1} & \textbf{1} & \textbf{1} & \textbf{1} & \textbf{1} & \textbf{1} \\
10 & 9 & 3 & 3 & 5 & \textbf{1} & \textbf{1} & \textbf{1} & \textbf{1} & \textbf{1} & \textbf{1} & \textbf{1} & \textbf{1} \\
12 & 11 & 3 & 3 & 3 & \textbf{1} & \textbf{1} & \textbf{1} & \textbf{1} & \textbf{1} & \textbf{1} & \textbf{1} & \textbf{1} \\
14 & 13 & 5 & 5 & 5 & N/A & N/A & 4 & N/A & \textbf{1} & \textbf{1} & \textbf{1} & \textbf{1} \\
16 & 15 & N/A & N/A & 5 & N/A & N/A & N/A & N/A & \textbf{1} & \textbf{1} & \textbf{1} & \textbf{1} \\
18 & N/A & N/A & N/A & N/A & N/A & N/A & N/A & N/A & \textbf{1} & \textbf{1} & \textbf{1} & \textbf{1} \\
20 & N/A & N/A & N/A & N/A & N/A & N/A & N/A & N/A & \textbf{1} & \textbf{1} & \textbf{1} & \textbf{1} \\
22 & N/A & N/A & N/A & N/A & N/A & N/A & N/A & N/A & N/A & \textbf{1} & \textbf{1} & \textbf{1} \\
\bottomrule
\end{tabular}%
}
\caption{Objective values using \textit{Default Search Strategy (Solver's Default)}}
\label{cp1-result}
\end{table}



\begin{table}[htbp]
\centering
\small
\resizebox{\textwidth}{!}{%
\begin{tabular}{c|cccc|cccc|cccc}
\toprule
\textbf{n} & \multicolumn{4}{c|}{\textbf{GECODE}} & \multicolumn{4}{c|}{\textbf{CHUFFED}} & \multicolumn{4}{c}{\textbf{CP-SAT}} \\
\cmidrule(lr){2-5}\cmidrule(lr){6-9}\cmidrule(lr){10-13}
  & bs & complete & noIMPL & noSB & bs & complete & noIMPL & noSB & bs & complete & noIMPL & noSB \\
\midrule
6 & \textbf{1} & \textbf{1} & \textbf{1} & \textbf{1} & \textbf{1} & \textbf{1} & \textbf{1} & \textbf{1} & \textbf{1} & \textbf{1} & \textbf{1} & \textbf{1} \\
8 & \textbf{1} & \textbf{1} & \textbf{1} & \textbf{1} & \textbf{1} & \textbf{1} & \textbf{1} & \textbf{1} & \textbf{1} & \textbf{1} & \textbf{1} & \textbf{1} \\
10 & \textbf{1} & \textbf{1} & \textbf{1} & \textbf{1} & \textbf{1} & \textbf{1} & \textbf{1} & \textbf{1} & \textbf{1} & \textbf{1} & \textbf{1} & \textbf{1} \\
12 & \textbf{1} & \textbf{1} & \textbf{1} & \textbf{1} & \textbf{1} & \textbf{1} & \textbf{1} & \textbf{1} & \textbf{1} & \textbf{1} & \textbf{1} & \textbf{1} \\
14 & \textbf{1} & \textbf{1} & \textbf{1} & \textbf{1} & N/A & \textbf{1} & \textbf{1} & N/A & N/A & N/A & N/A & N/A \\
16 & 15 & 3 & \textbf{1} & \textbf{1} & N/A & N/A & N/A & N/A & N/A & N/A & N/A & N/A \\
18 & N/A & N/A & N/A & N/A & N/A & N/A & N/A & N/A & N/A & N/A & N/A & N/A \\
20 & N/A & N/A & N/A & N/A & N/A & N/A & N/A & N/A & N/A & N/A & N/A & N/A \\
22 & N/A & N/A & N/A & N/A & N/A & N/A & N/A & N/A & N/A & N/A & N/A & N/A \\
\bottomrule
\end{tabular}%
}
\caption{Objective values using \textit{Sequential Custom Search Strategy}}
\label{cp2-result}
\end{table}

\begin{table}[htbp]
\centering
\small
\resizebox{\textwidth}{!}{%
\begin{tabular}{c|cccc|cccc|cccc}
\toprule
\textbf{n} & \multicolumn{4}{c|}{\textbf{GECODE}} & \multicolumn{4}{c|}{\textbf{CHUFFED}} & \multicolumn{4}{c}{\textbf{CP-SAT}} \\
\cmidrule(lr){2-5}\cmidrule(lr){6-9}\cmidrule(lr){10-13}
  & bs & complete & noIMPL & noSB & bs & complete & noIMPL & noSB & bs & complete & noIMPL & noSB \\
\midrule
6 & \textbf{1} & \textbf{1} & \textbf{1} & \textbf{1} & \textbf{1} & \textbf{1} & \textbf{1} & \textbf{1} & \textbf{1} & \textbf{1} & \textbf{1} & \textbf{1} \\
8 & \textbf{1} & \textbf{1} & \textbf{1} & \textbf{1} & \textbf{1} & \textbf{1} & \textbf{1} & \textbf{1} & \textbf{1} & \textbf{1} & \textbf{1} & \textbf{1} \\
10 & \textbf{1} & \textbf{1} & \textbf{1} & \textbf{1} & \textbf{1} & \textbf{1} & \textbf{1} & \textbf{1} & \textbf{1} & \textbf{1} & \textbf{1} & \textbf{1} \\
12 & \textbf{1} & \textbf{1} & \textbf{1} & \textbf{1} & \textbf{1} & \textbf{1} & \textbf{1} & \textbf{1} & \textbf{1} & \textbf{1} & \textbf{1} & \textbf{1} \\
14 & \textbf{1} & \textbf{1} & \textbf{1} & \textbf{1} & N/A & N/A & \textbf{1} & N/A & N/A & N/A & N/A & N/A \\
16 & \textbf{1} & \textbf{1} & \textbf{1} & \textbf{1} & N/A & N/A & 13 & N/A & N/A & N/A & N/A & N/A \\
18 & N/A & \textbf{1} & N/A & \textbf{1} & N/A & N/A & N/A & N/A & N/A & N/A & N/A & N/A \\
20 & N/A & N/A & N/A & N/A & N/A & N/A & N/A & N/A & N/A & N/A & N/A & N/A \\
22 & N/A & N/A & N/A & N/A & N/A & N/A & N/A & N/A & N/A & N/A & N/A & N/A \\
\bottomrule
\end{tabular}%
}
\caption{Objective values using \textit{Relax-and-Reconstruct (LNS) Strategy}}
\label{cp3-result}
\end{table}

The experimental results, Tables~\ref{cp1-result}~\ref{cp2-result}~\ref{cp3-result}, show that the \texttt{CP-SAT} solver, with its default search strategy, consistently delivers the best performance. 
Moreover, combining Sequential Search with the Relax-and-Reconstruct (LNS) strategy improves \texttt{Gecode} by solving larger instances and finding valid or optimal solutions where the default did not succeed.

\newpage

\section{SAT Model}
% This part is mandatory for groups of 4 students. Groups up to 3 students can choose between SAT and SMT, and will be given bonus points if they do both.

\subsection{Decision variables}
% Describe all the literals of your model and their semantics. For example, $\Delta_{i,j} = \text{true}$ iff the driver goes from city $i$ to city $j$.
Let $rb$ be the round robin tournament, $sts$ a schedule satisfying to the STS problem, and P, W, T the set of periods, weeks and teams respectively.
The satisfiability task is formalized by the following categories of propositions:
\begin{itemize}
    \item $matches\_schedule_{p, w, m} \leftrightarrow$ match $m$, denoting $rb_{m, w} = (t_1, t_2)$, takes place in period $p$ of week $w$
    \item $matches\_to\_periods_{t_1, t_2, p} \leftrightarrow$ $t_1$ plays against $t_2$ in period $p$
\end{itemize}
The optimization task, on the other hand, is expressed as follows.
\begin{itemize}
    \item $slots\_schedule_{p, w} \leftrightarrow$ team $sts_{p, w, 1} = t_1$ plays away
    \item $matches\_to\_slots_{t_1, t_2} \leftrightarrow$ $t_1$ plays away and $t_2$ plays at home
\end{itemize}

\subsection{Objective function}
Similarly to the other approaches, the objective is to minimize the absolute difference between the number of games played at home and away for each team.
Let $T$ be the set of teams and $A_t$ the number of times the team $t \in T$ plays away, the objective function translates into the following:
$$
k * = \underset{k \in \mathbb{N}}{\text{argmax}} \left( \forall t \in T, A_t \geq k \right)
$$
The chosen CP model allows to solve the optimization task independetly of the STS constraints using the $sts$ schedule computed in the satisfiability process. The optimization consists of a binary search of $k*$ in the interval $[1, \lfloor\frac{N-1}{2}\rfloor]$. Since SAT doesn't directly support optimization, a new set of optimizing constraints is introduced for every instance of $k$ .

The result is encoded in $slots\_schedule_{p, w}$, which cannot express the constraints alone, due to structural limitations of the encoding.
Instead \\$matches\_to\_slots_{t_1, t_2}$ is used: given a week $w$ and a period $p$, the team $sts_{p, w, 1} = t_1$ plays away if, and only if, the team $t_2 = sts_{p, w, 2}$ plays at home. Therefore, by definition:
$$
    \forall p \in P, w \in W.(slots\_schedule_{p, w} \leftrightarrow matches\_to\_slots_{sts_{p, w, 1}, sts_{p, w, 2}})
$$
Finally the optimization process is implemented by ensuring that for an instance $k \in [1, \lfloor\frac{N-1}{2}\rfloor]$:
$$
    \forall p \in P, t_1 \in T.(AtLeastK(matches\_to\_slots_{t_1, t_2} | t_2 \in T/\{t_1\}))
$$

\subsection{Constraints}
% Describe all the clauses of your model and their semantics. In particular, describe the encoding(s) that you used. Follow the indications given in Section 2.3 for main problem constraints, implied constraints, and symmetry breaking constraints.
\subsubsection{Every team plays once a week}
In a Round-Robin tournament $rb$, each team plays exactly once a week. 
Therefore the only problem is to ensure that for each week $w$, every match $rb_{m, w} = (t_i, t_j)$ is assigned to exactly one period $p$.

We enforce that every match is scheduled once:
$$
    \forall p \in P, w \in W.(ExactlyOne(matches\_schedule_{p, w, m} | m \in P))
$$
Finally no two matches are scheduled in the same period. 

We ensure that each match is scheduled to a unique period $p$ in the week $w$.
$$
    \forall m \in P, w \in W. ExactlyOne(matches\_schedule_{p, w, m} | p \in P)
$$
\subsubsection{Every team plays at most twice in the same period}
The constraint cannot be expressed using directly $matches\_schedule_{p, w, m}$, due to structural limitations of the encoding. 
Instead, we introduce the literal $matches\_to\_periods_{t_i, t_j, p}$: given a week $w$, the match $rb_{m,w}$ is scheduled in period $p$ if, and only if, the match $(rb_{m, w, 1}, rb_{m, w, 2}) = (t_1, t_2)$ takes place in period $p$. Therefore, by definition: 
$$
    \forall p \in P, w \in W, m \in P.matches\_schedule_{p, w, m} \leftrightarrow matches\_to\_periods_{rb_{m, w, 1}, rb_{m, w, 2}, p}
$$
Computationally this constraint soundly maps $matches\_schedule$ to \\$matches\_to\_periods$, which allow to express the main constraint:
$$
    \forall p \in P, t_1 \in T. AtMost2(matches\_to\_periods_{t_1, t_2, p} | t_2 \in T/\{t_1\})
$$
\subsection{Validation}
% See Section 2.4. The model \textbf{must} be implemented using at least Z3. Bonus points will be considered if a solver-independent language (e.g., Dimacs) is employed so as to play with different SAT solvers on the same model.
\subsubsection{Experimental design}
The model was written in Python by making use of the Z3 and the CVC5 library, which offers CaDiCaL and MiniSat as the underlying SAT solvers. The time elapsed to find an optimal solution, within the 300 second time limit, was measured and results presented in Fig.\ref{fig:SAT-result} . 
\subsubsection{Experimental results}
All solvers are able to find the optimal solution, which was much easier to find once the satisfying one was computed, but overall Z3 had the best performance in time and maximal size of the problem, i.e. $N=20$. On the other hand CaDiCaL was the worst, failing at $N=12$, while MiniSat stopped at $N=14$
\begin{figure}
    \centering
    \includegraphics[width=0.8\linewidth]{img/SAT-result.png}
    \caption{SAT optimization}
    \label{fig:SAT-result}
\end{figure}

\section{SMT Model}

Similarly to the SAT model, the SMT model is structured based on its two tasks: \textbf{Satisfiability} and \textbf{Optimization}.
We encode both the satisfiability and optimization phases in QF\_LIA (Quantifier-Free Linear Integer Arithmetic) because it lets us combine Boolean decisions with integer sums and comparisons directly, avoiding complex encodings of counters in pure SAT.

\subsection{Decision Variables}

\subsubsection{Satisfiability}

The decision variables for this task are:
\begin{itemize}
    \item $match\_schedule_{p,w,m} \in \{\text{true}, \text{false}\}$: true if, and only if, the match $m \in M$ is scheduled for period $p \in P$ of week $w \in W$.
    \item $home_{p,w} = t \in T$: if t is the home team in the match in period $p \in P$ of week $w \in W$.
    \item $away_{p,w} = t \in T$: if t is the away team in the match in period $p \in P$ of week $w \in W$.
\end{itemize}

\subsubsection{Optimization}
To optimize the balance of the number of games played at home and away for each team, the following variables are utilized:
\begin{itemize}
    \item $flip\_slot_{p,w} \in \{\text{true}, \text{false}\}$: % similarly to other models, it defines which team plays at home and which away in the matchup $(t_1, t_2)$ in week w and period p. 
    if the value is false, $rb_{p, w, 0}=t_1$ plays at home; otherwise it is true, then $rb_{p, w, 1}=t_2$ is the one playing at home.
    
    \item $home\_eff_{p,w},\, away\_eff_{p,w} \in T$: are, respectively, the effective home and effective away teams after the slot flip decisions.
\end{itemize}

\subsection{Objective Function}

The optimization goal is to minimize the maximum imbalance $k$ between the number of home and away games played by each team, respectively $H_t$ and $A_t$, throughout the tournament. Formally,

\[
k^* = \underset{k \in \mathbb{N}}{\text{argmin}} \left( \forall t \in T. |H_t - A_t| \leq k \right)
\]

where:
\[
H_t = \sum_{p,w} \mathbbm{1}[\,home\_eff_{p,w} = t\,], 
\quad
A_t = \sum_{p,w} \mathbbm{1}[\,away\_eff_{p,w} = t\,].
\]
The effective home and away variables are computed by considering slot flips:

\begin{align*}
home\_eff_{p,w} =& \text{ite}(flip\_slot_{p,w},\, away_{p,w},\, home_{p,w}), \\
away\_eff_{p,w} =& \text{ite}(flip\_slot_{p,w},\, home_{p,w},\, away_{p,w}).
\end{align*}
where ite stands for "if then else" statement.

\subsection{Constraints}

\subsubsection{Each match is assigned to a unique period each week}

Each period in each week is assigned to exactly one match:

\[
\forall p \in P,\, w \in W: 
\sum_{m \in M} match\_schedule_{p,w,m} = 1.
\]
Each match is assigned to exactly one period each week:

\[
\forall m \in M,\, w \in W: 
\sum_{p \in P} match\_schedule_{p,w,m} = 1.
\]

\subsubsection{Binding team assignments}
The teams playing in $match\_schedule_{p, w, m}$ are linked to the corresponding variables $home_{p, w}$ and $away_{p, w}$ based on the values of $rb_{m, w}$ as follows:

$$
\forall p \in P,\, w \in W.(match\_schedule_{p,w,m} \leftrightarrow (home_{p,w} = rb_{m,w,0} \ \wedge \ away_{p,w} = rb_{m,w,1})
\big)
$$

\subsubsection{Every team plays at most twice in the same period}

For each team $t$ and each period $p$, we count how many times $t$ appears either as home or away in that period over all weeks and constraint the count to not exceed 2:

\[
\forall t \in T,\, p \in P: 
\sum_{w \in W} \sum_{m \in M} 
\big(
[\,rb_{m,w,0} = t \lor rb_{m,w,1} = t\,] \cdot [\,match\_schedule_{p,w,m}\,]
\big) \leq 2.
\]

\subsubsection{Symmetry breaking}

To reduce equivalent permutations of the solutions, the first match is fixed:

\[
match\_schedule_{0,0,0} = \text{true}.
\]

\subsection{Validation}

\subsubsection{Experimental Design}

Solver performance is measured in two stages:
\begin{enumerate}
  \item \textbf{Satisfiability}: Produce a single \texttt{.smt2} file that encodes match assignments, period-limit, and symmetry breaking constraints as linear arithmetic formulas. The solver then finds an initial feasible schedule.
  \item \textbf{Optimization}: Perform a binary search over 
  \[
    k \in \{1,2,\dots,N/2\}.
  \]
  For each candidate \(k\), generate a new \texttt{.smt2} file that
  \begin{itemize}
    \item has the base \(home/away\) assignments,
    \item adds boolean \texttt{flip\_slot} variables,
    \item defines effective \(home\_eff,away\_eff\) via \texttt{ite},
    \item asserts \(\lvert H_t - A_t\rvert \le k\) using integer counters \(H_t,A_t\).
  \end{itemize}
  This produces \(\log_2(N) - 1\) optimization files.
\end{enumerate}

The total runtime is measured from the start of the satisfiability phase until either a solution for \(k = 1\) is found or the 300\,s timeout is reached.
We tested the following solvers: Z3 4.15.1.0 and CVC5 1.3.0 and present our findings in table \ref{table:smt-result}. 

\subsubsection{Experimental results}

The experimental results show that Z3 performs significantly better than CVC5 in terms of solving time and scalability. Specifically, Z3 solved instances up to $N=22$, while CVC5 handled only smaller instances up to $N=8$.

\begin{table}[H]
\centering
\small
\begin{tabular}{|c|c|c|}
\toprule
\textbf{N} & \textbf{Z3} & \textbf{CVC5} \\
\midrule
6  & \textbf{1}   & \textbf{1}   \\
8  & \textbf{1}   & \textbf{1}   \\
10 & \textbf{1}   & N/A \\
12 & \textbf{1}   & N/A \\
14 & \textbf{1}   & N/A \\
16 & \textbf{1}   & N/A \\
18 & \textbf{1}   & N/A \\
20 & \textbf{1}  & N/A \\
22 & \textbf{1} & N/A \\
24 & N/A & N/A \\
\bottomrule
\end{tabular}
\caption{SMT optimization solver results}
\label{table:smt-result}
\end{table}

% \begin{figure}[H]
% \centering
% \includegraphics[width=0.8\linewidth]{img/SMT-result.png}
% \caption{SMT solution example}
% \label{fig:smt-result}
% \end{figure}


\section{MIP}
\subsection{Decision variables}
\subsubsection{matches}
The binary decision variables $X_{wpm}$ are equal to 1 iff in week w and period p match m is played.

\subsubsection{slots}
The binary decision variables $A_{wp}$ determine which team plays at home and which away. A is indexed by weeks and periods, so $A_wp$ corresponds to the matchup between teams $t_1, t_2$ in week w and period p; if $A_{wp} = 0$ then $t_1$ plays at home and $t_2$ away, if, otherwise, $A_{wp} = 1$ the order is reversed.

\subsubsection{team periods of play}
The binary variables TP are used to constrain each team to playing at most twice in the same period. $TP_{twp} = 1$ iff team t plays in week w and period p. 

\subsubsection{home games counter}
To compute the common objective function in MIP, it is necessary to introduce an array of integer auxiliary variables H such that $H[t]$ is the number of home games team t plays, bounded in $[0, n-1]$.

\subsection{Objective variables}
To compute D as the maximum difference between games played at home and away for all teams, we first find for each team the number of games played at home (2), and then constraint D as being greater or equal than the difference of home and away games for each team, given that we have a minimization problem this is effectively equivalent to computing the max. The absolute value of the difference is not computed explicitly but decomposed into 2 inequalities (3)(4). Finally, we look for the minimum of D (1)

\begin{align}
    &\min  D \\
    H_t =& \sum_{w, p} (A_{wp} = 0 \land G_{wp}[\text{home}] = t) \notag \\
    +& \sum_{w, p} (A_{wp} = 1 \land G_{wp}[\text{away}] = t) \hspace{20px}  &&t =1, \dots, n\\
    D &\geq 2H_t - (n-1) && t = 1, \dots, n\\
    D & \geq -(2 H_t - (n-1)) && t = 1, \dots, n
\end{align}


\subsection{Constraints}
\subsubsection{periods and matches}
Due to how the decision variables are defined, it was necessary to impose that each period in each week is assigned a single match (5) and each match is assigned to a single period (6).

\begin{align}
    \sum_{m = 1}^{n/2} X_{wpm} &= 1 \hspace{20px} p =1, \dots, n/2 \hspace{10px} w = 1,\dots, n-1 \\
    \sum_{p = 1}^{n/2} X_{wpm} &= 1 \hspace{20px} m = 1, \dots, n/2 \hspace{10px} w = 1,\dots, n-1
\end{align}

\subsubsection{team playing at most twice in the same period}
The constraint on teams playing at most twice in the same period was imposed by first linking the variables of X to those in TP based on the values in G (7) and then imposing that the sum of periods of play is smaller than 2 (8). 

\begin{align}
    TP_{twp} &= X_{wpm} \hspace{20px} G_{wm} =(t1, t2) \land (t = t1 \lor t=t2) \hspace{20px} \forall t \forall p  \forall w\\
    \sum_{w=1}^{n-1} &TP_{twp} \leq 2 \hspace{40px} t=1, \dots, n \hspace{10px} p = 1, \dots, n/2
\end{align}

\subsection{Validation}
\subsubsection*{Experimental design}
The model was written in Python by making use of the PuLP library and the solvers tested on the MIP model were: CBC 2.10.3, HiGHS 1.10.0, CPLEX 22.1.1 and SCIP 5.5.0 with their default parameters. The time elapsed to find an optimal solution, within the 300 second time limit, was measured and results presented in Fig.\ref{fig:MIP-solution} . All tests were run on a single core of an Intel i7-10750H CPU.

\subsubsection*{Experimental results}
As shown in Fig. \ref{fig:MIP-solution} CBC had the worst performance, it isn't able to find the optimal solution for n greater than 14. CPLEX, instead, was the fastest up to n=14 but after n=16 it stopped finding the optimal solution. HiGHS was able to find an optimal solution up to n=18 and SCIP, the best performer of the four, was able to reach n=20.

It was also verified that the solvers either found and optimal solution or no solution at all.

\begin{figure}
    \centering
    \includegraphics[width=0.8\linewidth]{img/MIP-result.png}
    \caption{MIP optimization}
    \label{fig:MIP-solution}
\end{figure}

\bibliographystyle{unsrt}
\bibliography{ref.bib}

\end{document}
