\section{Conclusions}
This report explored the Sports Tournament Scheduling (STS) problem with CP, SAT, SMT and MIP. Across all approaches, the precomputation of the round-robin tournament matrix significantly sped up the search process by reducing the search space and ensuring fundamental schedule properties.

While CP, SAT, SMT, and MIP models generally yielded comparable results, CP and SMT demonstrated slightly better performance. OR-Tools CP-SAT, for the  CP model, and Z3, for the SMT model, were able to solve instances up to N=22 to optimality within the time limits. In contrast, the best performing SAT solver, Z3, reached N=20, and the best MIP solver, SCIP, also reached N=20.

The experimental results presented in this report were conducted under single-threaded conditions. Therefore, a promising avenue for further performance improvement would be to explore the benefits of parallelism, leveraging multi-threading capabilities to potentially solve larger instances or achieve faster optimal solutions.

\noindent\textbf{Authenticity and Author Contribution Statement}
We declare that the work presented in this report is our own and has not been copied from any external source, except where explicitly cited. All modelling decisions, encodings, implementations and experiments were carried out by us without unauthorized assistance.
All the modelling decisions were agreed on as a group, while the implementations were carried out more autonomously:
\begin{itemize}
    \item Cono Cirone: CP
    \item Giulio Petrozziello: SAT
    \item Jacopo Dardini: SMT
    \item Gio Formichella: MIP
\end{itemize}