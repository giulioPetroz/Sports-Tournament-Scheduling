\section{CP Model}
\subsection{Decision Variables}
This model utilizes two primary decision variables to construct the tournament schedule and assign home/away teams for each match.

\subsubsection{\text{matches}}
% For $p \in P, w \in W$, $matches_{p, w} = m \in M$ determines that $rb_{m, w}$ is scheduled in week $w$ and period $p$.

For each period $p \in P$ and week $w \in W$
$$
matches_{p,w}=m \in M
$$
determines that $rb_{m, w}$ is scheduled in week $w$ and period $p$.


\subsubsection{\text{home\_away}}
For each $p \in P$ and week $w \in W$, $home\_away_{p, w} \in \{0, 1\}$, assigns the home team for the match scheduled at period $p$ in week $w$. Specifically, $\text{home\_away}_{p, w} = 0$ means the first team listed in the match plays at home, while $\text{home\_away}_{p, w} = 1$ means the second team plays at home.

\subsection{Auxiliary Variables}
\subsubsection{\text{home\_games}}
The auxiliary variable $\text{home\_games}_{t} \in [1, \dots, N-1]$ for $t \in T$, represents the total number of home games assigned to team $t$ throughout the tournament. Its value is derived from the assignments of the decision variables $\text{matches}$ and $\text{home\_away}$.


\subsection{Objective Function}

The model's objective is to quantify and minimize disparities in home game assignments through the $\text{max\_imbalance}$ variable.

\subsubsection{Objective Variable: \text{max\_imbalance}}\label{objective}
The integer variable $\text{max\_imbalance} \in [1, \dots, N-1]$ quantifies the maximum absolute disparity in home game assignments across all teams. The lower bound of 1 acknowledges that perfect balance ($\text{max\_imbalance}=0$) is not achievable for an odd number of total games ($N-1$ games). The upper bound of $N-1$ represents the theoretical maximum possible deviation, occurring if a team plays all its games either at home or away.

The value of $\text{max\_imbalance}$ is determined by the following fairness constraint:
\[ \forall t \in T : \left| 2 \times \text{home\_games}_{t} - (N-1) \right| \leq \text{max\_imbalance} \]
This formulation precisely defines the absolute "imbalance" for each team $t$. This imbalance is derived from the difference between a team's home games and away games. Since $\text{home\_games}_{t} + \text{away\_games}_{t} = N-1$ (total games), substituting the expression for $\text{away\_games}_{t}$ yields the imbalance for team $t$ as $2 \times \text{home\_games}_{t} - (N-1)$. By enforcing that the absolute value of this imbalance for every team must be less than or equal to $\text{max\_imbalance}$, this variable effectively captures the largest such deviation among all teams, serving as the direct measure of the overall schedule's fairness.


\subsubsection{Objective}
The objective is to \textbf{minimize $\text{max\_imbalance}$}:
\[ \text{minimize } \text{max\_imbalance} \]
This aims to achieve the fairest possible distribution of home and away games. An optimal solution would ideally yield $\text{max\_imbalance}=1$, meaning each team's home/away game count differs by at most one, which is the best possible outcome for an odd number of total games played.


\subsection{Constraints}
\subsubsection{Core Constraints}
These constraints are strictly necessary for defining a feasible round-robin schedule:
\begin{enumerate}
    \item \textbf{Each period must be used exactly once per week:} Ensures that for every week, all matches generated by the round-robin structure's periods are indeed scheduled. Without this, some pairings might be missed, or periods might be duplicated, leading to an incomplete or invalid schedule.
    \[ \forall w \in W : \texttt{all\_different}([\texttt{matches}[p, w] \mid p \in P]) \]

    \item \textbf{Each team plays at most twice per period:} 
    % TODO explain the implementation please!
\[ \forall p \in P, \forall t \in T : \left| \{ (w, s) \mid w \in W, s \in S, \text{rb}_{\text{matches}_{p, w}, w, s} = t \} \right| \leq 2 \]

\end{enumerate}

\subsubsection{Channeling Constraints}
\begin{enumerate}
 \item \textbf{Calculation of Home Games:} This constraint defines the value of $\text{home\_games}_{t}$.
\[ \forall t \in T : \text{home\_games}_{t} = \sum_{p \in P, w \in W, s \in S} \mathbb{I}\left( \text{rb}_{\text{matches}_{p, w}, w, s} = t \land \text{home\_away}_{p, w} = s \right) \]
The indicator function $\mathbb{I}(\cdot)$ ensures that 1 is added to the sum if team $t$ is located in slot $s$ and that slot $s$ is designated as the home slot by $\text{home\_away}_{p,w}$.
\end{enumerate}

\subsubsection{Implied Constraints}
% TODO: explain relationship with all_different
\begin{enumerate}
    % TODO explain the implementation please!
    \item \textbf{Each team appears exactly once per week:}
    \[ \forall w \in W, \forall t \in T : \left| \{ (p, s) \mid p \in P, s \in S, \text{rb}_{\text{matches}_{p, w}, w, s} = t \} \right| = 1 \]
\end{enumerate}

\subsubsection{Symmetry Breaking Constraints}

\begin{enumerate}
    \item \textbf{Break period assignment symmetry using lexicographic ordering:} The order in which matches corresponding to $P$ are assigned within $\text{matches}$ for each week is symmetrical. This constraint breaks such symmetries by enforcing a lexicographical ordering, reducing the number of equivalent search paths.
    \[ (\text{matches}_{p, w})_{p \in P, w \in W} \succeq_{\text{lex}} (\text{matches}_{p, w})_{\text{reversed}(p) \in P, w \in W} \]
    This states that the sequence of $\text{matches}$ variables, when read in normal $(p,w)$ order, must be lexicographically greater than or equal to when read in $(p,w)$ order with $p$ reversed. This helps to fix one permutation of period assignments.

    \item \textbf{Fix first match home assignment to break home/away symmetry:} This constraint eliminates global home/away assignment symmetry by fixing the home/away status of the first match.
    \[ \text{home\_away}_{0, 0} = 0 \]

% TODO: move to Optimization
\item \textbf{Balance the home/away assignments within each week:} While not a strict symmetry breaking constraint, this constraint helps reduce the search space by ensuring that within each week $w$, the number of matches where the second team plays at home ($\text{home\_away}_{p, w}=1$) is roughly half of the total matches ($|P|$), with a maximum deviation of 1:
\[ \forall w \in W : \left| \sum_{p \in P} \text{home\_away}_{p, w} - \left\lfloor \frac{|P|}{2} \right\rfloor \right| \leq 1 \]
This guides the solver towards balanced assignments and prunes highly imbalanced weekly configurations.

\end{enumerate}

\subsection{Validation}

The model was implemented in MiniZinc and validated through a series of experiments designed to assess solver performance under various model configurations and search strategies.

\subsubsection{Experimental Design}

To comprehensively evaluate the performance of different solving strategies for the Sports Tournament Scheduling problem, a systematic experimental study was conducted.

\textbf{Hardware and Software:}
Experiments were executed on a MacBook Air M1 equipped with an 8-core CPU. The following solvers were employed: \textit{Gecode}, \textit{Chuffed} and \textit{OR-Tools CP-SAT}. 
A uniform time limit of $300$ seconds was imposed for each individual problem instance.

\textbf{Model Configurations:}
Four configurations were tested: \texttt{baseline (core)}, \texttt{baseline+implied}, \texttt{baseline+symmetry breaking}, \texttt{full model}.

\textbf{Search Strategies:}
Three distinct search strategies were employed to analyze solver behavior, with a particular focus on how they influenced Gecode, often considered to have weaker default heuristics compared to modern SAT-based solvers.

\textbf{Search Strategies:}
Three distinct search strategies were employed to analyze solver behavior, focusing on their influence on Gecode, given its often weaker default heuristics compared to modern SAT-based solvers.

\begin{enumerate}
    \item \textbf{Default Search Strategy (Solver's Default):} Each solver relied entirely on its built-in decision heuristics and restart policies, serving as a baseline for their inherent capabilities.

    \item \textbf{Sequential Custom Search Strategy:} A manually defined sequential search (\texttt{seq\_search}) was applied, prioritizing \texttt{matches} variables with \texttt{dom\_w\_deg} and \texttt{home\_away} variables with \texttt{first\_fail}, utilizing a \texttt{restart\_luby(100)} policy.

    \item \textbf{Relax-and-Reconstruct (LNS) Strategy:} This higher-level strategy incorporated \texttt{relax\_and\_reconstruct} on the \texttt{matches} variables (preserving 60\% of solution values), leveraging Large Neighborhood Search (LNS) techniques. It was layered on top of the "Sequential Custom Search Strategy."
\end{enumerate}

\textbf{Solver-Specific Strategy Application:}
To ensure a fair and controlled comparison under single-threaded conditions (aligning with project constraints), OR-Tools CP-SAT was run without multi-threading. For both Chuffed and OR-Tools CP-SAT, the \texttt{free\_search} parameter was explicitly omitted when applying the custom Sequential Custom Search and Relax-and-Reconstruct strategies. This allowed direct evaluation of the user-defined MiniZinc search annotations, rather than the solvers' highly optimized default heuristics.


\subsubsection{Experimental Results}

\begin{table}[htbp]
\centering
\small
\resizebox{\textwidth}{!}{%
\begin{tabular}{c|cccc|cccc|cccc}
\toprule
\textbf{n} & \multicolumn{4}{c|}{\textbf{GECODE}} & \multicolumn{4}{c|}{\textbf{CHUFFED}} & \multicolumn{4}{c}{\textbf{CP-SAT}} \\
\cmidrule(lr){2-5}\cmidrule(lr){6-9}\cmidrule(lr){10-13}
  & bs & complete & noIMPL & noSB & bs & complete & noIMPL & noSB & bs & complete & noIMPL & noSB \\
\midrule
6 & 0 & 0 & 0 & 0 & 0 & 0 & 0 & 0 & 0 & 0 & 0 & 0 \\
8 & 6 & 0 & 0 & 7 & 0 & 0 & 0 & 0 & 0 & 0 & 0 & 0 \\
10 & N/A & N/A & 0 & N/A & 0 & 0 & 0 & 0 & 1 & 1 & 0 & 1 \\
12 & N/A & N/A & 0 & N/A & 5 & 1 & 3 & 2 & 3 & 2 & 2 & 2 \\
14 & N/A & N/A & N/A & N/A & 180 & 21 & 69 & 141 & 5 & 6 & 5 & 5 \\
16 & N/A & N/A & N/A & N/A & N/A & N/A & N/A & N/A & 16 & 29 & 29 & 13 \\
18 & N/A & N/A & N/A & N/A & N/A & N/A & N/A & N/A & 265 & 35 & 39 & 270 \\
20 & N/A & N/A & N/A & N/A & N/A & N/A & N/A & N/A & 84 & 67 & 66 & 82 \\
22 & N/A & N/A & N/A & N/A & N/A & N/A & N/A & N/A & 238 & 157 & 118 & 253 \\
\bottomrule
\end{tabular}%
}
\caption{CPU time in seconds for finding the \textit{optimal solution} using \textit{Default Search Strategy (Solver's Default)}}
\end{table}

\begin{table}[htbp]
\centering
\small
\resizebox{\textwidth}{!}{%
\begin{tabular}{c|cccc|cccc|cccc}
\toprule
\textbf{n} & \multicolumn{4}{c|}{\textbf{GECODE}} & \multicolumn{4}{c|}{\textbf{CHUFFED}} & \multicolumn{4}{c}{\textbf{CP-SAT}} \\
\cmidrule(lr){2-5}\cmidrule(lr){6-9}\cmidrule(lr){10-13}
  & bs & complete & noIMPL & noSB & bs & complete & noIMPL & noSB & bs & complete & noIMPL & noSB \\
\midrule
6 & 0 & 0 & 0 & 0 & 0 & 0 & 0 & 0 & 0 & 0 & 0 & 0 \\
8 & 0 & 0 & 0 & 0 & 0 & 0 & 0 & 0 & 0 & 0 & 0 & 0 \\
10 & 0 & 0 & 0 & 0 & 0 & 0 & 0 & 0 & 1 & 1 & 1 & 1 \\
12 & 0 & 0 & 0 & 0 & 0 & 0 & 0 & 0 & 54 & 55 & 58 & 47 \\
14 & 4 & 8 & 4 & 7 & N/A & 57 & 34 & N/A & N/A & N/A & N/A & N/A \\
16 & N/A & N/A & 191 & 125 & N/A & N/A & N/A & N/A & N/A & N/A & N/A & N/A \\
\bottomrule
\end{tabular}%
}
\caption{CPU time in seconds for finding the \textit{optimal solution} using \textit{Sequential Custom Search Strategy}}
\end{table}

\begin{table}[htbp]
\centering
\small
\resizebox{\textwidth}{!}{%
\begin{tabular}{c|cccc|cccc|cccc}
\toprule
\textbf{n} & \multicolumn{4}{c|}{\textbf{GECODE}} & \multicolumn{4}{c|}{\textbf{CHUFFED}} & \multicolumn{4}{c}{\textbf{CP-SAT}} \\
\cmidrule(lr){2-5}\cmidrule(lr){6-9}\cmidrule(lr){10-13}
  & bs & complete & noIMPL & noSB & bs & complete & noIMPL & noSB & bs & complete & noIMPL & noSB \\
\midrule
6 & 0 & 0 & 0 & 0 & 0 & 0 & 0 & 0 & 0 & 0 & 0 & 0 \\
8 & 0 & 0 & 0 & 0 & 0 & 0 & 0 & 0 & 0 & 0 & 0 & 0 \\
10 & 0 & 0 & 0 & 0 & 0 & 0 & 0 & 0 & 1 & 1 & 1 & 1 \\
12 & 0 & 0 & 0 & 0 & 8 & 96 & 3 & 2 & 54 & 63 & 59 & 50 \\
14 & 1 & 5 & 1 & 0 & N/A & N/A & 184 & 250 & N/A & N/A & N/A & N/A \\
16 & 183 & 19 & 1 & 8 & N/A & N/A & N/A & N/A & N/A & N/A & N/A & N/A \\
18 & N/A & 3 & N/A & N/A & N/A & N/A & N/A & N/A & N/A & N/A & N/A & N/A \\
\bottomrule
\end{tabular}%
}
\caption{CPU time in seconds for finding the \textit{optimal solution} using \textit{Relax-and-Reconstruct (LNS) Strategy}}
\end{table}