\section{CP Model}
\subsection{Decision Variables}
This model utilizes two primary decision variable matrices to construct the tournament schedule and assign home/away teams for each match.

\subsubsection{\text{matches}}
For each period $p \in P$ and week $w \in W$
$$
matches_{p,w}=m \in M
$$
determines that $rb_{m, w}$ is scheduled in week $w$ and period $p$.


\subsubsection{\text{flip\_slot}}  
For each period $p \in P$ and week $w \in W$  
$$  
flip\_slot_{p,w} = a \in \{0, 1\}  
$$  
determines which team is assigned as the home team for the match $(t_1, t_2)$ scheduled in week $w$ and period $p$. Specifically, $a = 0$ means that $t_1$ plays at home while $t_2$ away; on the contrary, $a = 1$ means $t_2$ plays at home and $t_1$ away.


\subsection{Auxiliary Variables}
\subsubsection{\text{home\_games}}  
For each team $t \in T$  
$$  
home\_games_{t} = h \in [0, \dots, N-1]  
$$  
determines the total number of home games assigned to team $t$ throughout the tournament. Its value is derived from the assignments of the decision variables $matches$ and $flip\_slot$.


\subsection{Objective Function}

The model's objective is to minimize disparities in the number of home and away games through the $\text{max\_imbalance}$ variable.

\subsubsection{Objective Variable: \text{max\_imbalance}}\label{objective}
The integer variable $\text{max\_imbalance} \in [1, \dots, N-1]$ quantifies the maximum absolute disparity in home and away game assignments across all teams. The lower bound of 1 is the best possible balance score, $\text{max\_imbalance}=0$ is not achievable for an odd number of games, $N-1$. The upper bound of $N-1$ represents the theoretical maximum possible deviation, occurring if a team plays all its games either at home or away.

The value of $\text{max\_imbalance}$ is determined by looking for the minimum integer satisfying the following fairness constraints:
\[ \forall t \in T : \left| 2 \times \text{home\_games}_{t} - (N-1) \right| \leq \text{max\_imbalance} \]
This formulation is derived from the difference between a team's home games and away games: $|\text{home\_games}_t - \text{away\_games}_t|$. Since \\ $\text{home\_games}_{t} + \text{away\_games}_{t} = N-1$ (total games), expressing $\text{away\_games}_t$ as a function of the other quantities and substituting it in the starting expression, yields $2 \times \text{home\_games}_{t} - (N-1)$. By enforcing that this score must be less than or equal to $\text{max\_imbalance}$ for every team, we constrain the variable to being greater than the maximum.

The matchups inside $rb$ have an unbalanced structure: the team with the smaller identifier is always in the first slot, and the team with the larger identifier in the second. To guide the search towards more balanced solutions, we introduce the following constraint:
\[
\forall w \in W : \left| \sum_{p \in P} \text{flip\_slot}_{p, w} - \left\lfloor \frac{|P|}{2} \right\rfloor \right| \leq 1.
\]
This condition ensures that within each week, the number of matches where the second team plays at home is roughly half of the total matches. By balancing the use of flip\_slot values, the solver is guided to assign home games more equitably, ensuring that teams with smaller IDs can play away and teams with higher IDs can play at home. This contributes to balancing the number of home and away games for each team, ultimately speeding up the search for an optimal solution.

\subsubsection{Objective}
The objective is to \textbf{minimize $\text{max\_imbalance}$}:
\[ \text{minimize } \text{max\_imbalance} \]
The aim is to achieve the fairest possible distribution of home and away games; an optimal solution has objective value $\text{max\_imbalance}=1$.

\subsection{Constraints}
\subsubsection{Core Constraints}
These constraints are strictly necessary for defining a feasible round-robin schedule:
\begin{enumerate}
    \item \textbf{Each match is assigned to a unique period each week:} Ensures that for every week, all matches generated by the round-robin structure's periods are indeed scheduled. Without this, some pairings might be missed, or periods might be duplicated, leading to an incomplete or invalid schedule.
    \[ \forall w \in W : \texttt{all\_different}([\texttt{matches}[p, w] \mid p \in P]) \]

    \item \textbf{Each team plays at most twice in the same period:} 
\[ \forall p \in P, \forall t \in T : \left| \{ (w, s) \mid w \in W, s \in S, \text{rb}_{\text{matches}_{p, w}, w, s} = t \} \right| \leq 2 \]
Implemented using the \texttt{global\_cardinality} global constraint.

\end{enumerate}

\subsubsection{Channeling Constraints}
\begin{enumerate}
 \item \textbf{Home games:} This constraint defines the value of $\text{home\_games}_{t}$.
\[ \forall t \in T : \text{home\_games}_{t} = \sum_{p \in P, w \in W, s \in S} \mathbb{I}\left( \text{rb}_{\text{matches}_{p, w}, w, s} = t \land \text{flip\_slot}_{p, w} = s \right) \]
The indicator function $\mathbb{I}(\cdot)$ ensures that 1 is added to the sum if team $t$ is located in slot $s$ and that slot $s$ is designated as the home slot by $\text{flip\_slot}_{p,w}$.
\end{enumerate}


\subsubsection{Implied Constraints}
\begin{enumerate}
   \item \textbf{Each team appears exactly once per week:}  While this is implicitly ensured by the combination of the \texttt{all\_different} constraint on \texttt{matches} and the construction of \texttt{rb}, explicitly stating it helps the solver propagate information earlier and prune the search space more effectively.  It is implemented using the \texttt{global\_cardinality} global constraint:
    \[
     \forall w \in W, \forall t \in T : \left| \{ (p, s) \mid p \in P, s \in S, \text{rb}_{\text{matches}_{p, w}, w, s} = t \} \right| = 1.
 \]
\end{enumerate}



\subsubsection{Symmetry Breaking Constraints}

\begin{enumerate}
\item \textbf{Break period assignment symmetry with lexicographic ordering:}
The order in which match periods are assigned within each week is symmetric, meaning that swapping period labels across weeks yields equivalent schedules. To eliminate this redundancy, we enforce a lexicographic ordering:
$$
(\text{matches}_{p, w})_{p \in P, w \in W} \succ_{\text{lex}} (\text{matches}_{p, w})_{\text{reversed}(p) \in P, w \in W}
$$
This ensures that the sequence of $\text{matches}$ variables, when read in normal $(p,w)$ order, must be lexicographically greater than or equal to when read with $p$ reversed. Due to the \texttt{all\_different} constraint on $\text{matches}_{p,w}$ per week, this comparison typically resolves early by comparing $\text{matches}_{0,0}$ and $\text{matches}_{|P| - 1,0}$. However, we prefer the full lexicographic constraint over a simple pairwise comparison because it provides stronger constraint propagation improving solver performance overall.

    \item \textbf{Fix first match home} This constraint eliminates global home/away assignment symmetry by fixing the home/away status of the first match.
    \[ \text{flip\_slot}_{0, 0} = 0 \]


\end{enumerate}

\subsection{Validation}

\subsubsection{Experimental Design}

The model was implemented in MiniZinc and validated through a series of experiments designed to assess solver performance under various model configurations and search strategies.

\textbf{Hardware and Software:}
Experiments were executed on a MacBook Air M1 equipped with an 8-core CPU. The following solvers were employed: \textit{Gecode 6.3.0}, \textit{Chuffed 0.13.2} and \textit{OR-Tools CP-SAT 9.12.4544}. 
The same time limit of $300$ seconds was imposed for each individual problem instance.

\textbf{Model Configurations:}
Four configurations were tested: \texttt{baseline (core)}, \texttt{baseline+implied}, \texttt{baseline+symmetry breaking}, \texttt{full model}.

\textbf{Search Strategies:}
Three distinct search strategies were employed to analyze solver behavior, focusing on their influence on Gecode, given its often weaker default heuristics compared to modern SAT-based solvers.

\begin{enumerate}
    \item \textbf{Default Search Strategy (Solver's Default):} Each solver relied entirely on its built-in decision heuristics and restart policies, serving as a baseline for their inherent capabilities.

    \item \textbf{Sequential Custom Search Strategy:} A manually defined sequential search (\texttt{seq\_search}) was applied, prioritizing \texttt{matches} variables with \texttt{dom\_w\_deg} and \texttt{flip\_slot} variables with \texttt{first\_fail}, utilizing a \\ \texttt{restart\_luby(100)} policy.

    \item \textbf{Relax-and-Reconstruct (LNS) Strategy:} This higher-level strategy incorporated \texttt{relax\_and\_reconstruct} on the \texttt{matches} variables (preserving 60\% of solution values), leveraging Large Neighborhood Search (LNS) techniques. It was layered on top of the "Sequential Custom Search Strategy."
\end{enumerate}

\textbf{Solver-Specific Strategy Application:}
To ensure a fair and controlled comparison under single-threaded conditions (aligning with project constraints), OR-Tools CP-SAT was run without multi-threading. For both Chuffed and OR-Tools CP-SAT, the \texttt{free\_search} parameter was explicitly omitted when applying the custom Sequential Custom Search and Relax-and-Reconstruct strategies. This allowed direct evaluation of the user-defined MiniZinc search annotations, rather than the solvers' highly optimized default heuristics.


\subsubsection{Experimental Results}
In the following tables, we report CPU execution times for different solvers and the following four model configurations:
\begin{itemize}
    \item \texttt{bs} denotes the baseline (core) model
     \item  \texttt{noIMPL} denotes the baseline with symmetry breaking constraints
    \item \texttt{noSB} denotes the baseline with implied constraints
    \item \texttt{complete} denotes the full model with both implied and symmetry breaking constraints
\end{itemize}

\begin{table}[htbp]
\centering
\small
\resizebox{\textwidth}{!}{%
\begin{tabular}{c|cccc|cccc|cccc}
\toprule
\textbf{n} & \multicolumn{4}{c|}{\textbf{GECODE}} & \multicolumn{4}{c|}{\textbf{CHUFFED}} & \multicolumn{4}{c}{\textbf{CP-SAT}} \\
\cmidrule(lr){2-5}\cmidrule(lr){6-9}\cmidrule(lr){10-13}
  & bs & complete & noIMPL & noSB & bs & complete & noIMPL & noSB & bs & complete & noIMPL & noSB \\
\midrule
6 & \textbf{1} & \textbf{1} & \textbf{1} & \textbf{1} & \textbf{1} & \textbf{1} & \textbf{1} & \textbf{1} & \textbf{1} & \textbf{1} & \textbf{1} & \textbf{1} \\
8 & \textbf{1} & \textbf{1} & \textbf{1} & \textbf{1} & \textbf{1} & \textbf{1} & \textbf{1} & \textbf{1} & \textbf{1} & \textbf{1} & \textbf{1} & \textbf{1} \\
10 & 9 & 3 & 3 & 5 & \textbf{1} & \textbf{1} & \textbf{1} & \textbf{1} & \textbf{1} & \textbf{1} & \textbf{1} & \textbf{1} \\
12 & 11 & 3 & 3 & 3 & \textbf{1} & \textbf{1} & \textbf{1} & \textbf{1} & \textbf{1} & \textbf{1} & \textbf{1} & \textbf{1} \\
14 & 13 & 5 & 5 & 5 & N/A & N/A & 4 & N/A & \textbf{1} & \textbf{1} & \textbf{1} & \textbf{1} \\
16 & 15 & N/A & N/A & 5 & N/A & N/A & N/A & N/A & \textbf{1} & \textbf{1} & \textbf{1} & \textbf{1} \\
18 & N/A & N/A & N/A & N/A & N/A & N/A & N/A & N/A & \textbf{1} & \textbf{1} & \textbf{1} & \textbf{1} \\
20 & N/A & N/A & N/A & N/A & N/A & N/A & N/A & N/A & \textbf{1} & \textbf{1} & \textbf{1} & \textbf{1} \\
22 & N/A & N/A & N/A & N/A & N/A & N/A & N/A & N/A & N/A & \textbf{1} & \textbf{1} & \textbf{1} \\
\bottomrule
\end{tabular}%
}
\caption{Objective values using \textit{Default Search Strategy (Solver's Default)}}
\end{table}



\begin{table}[htbp]
\centering
\small
\resizebox{\textwidth}{!}{%
\begin{tabular}{c|cccc|cccc|cccc}
\toprule
\textbf{n} & \multicolumn{4}{c|}{\textbf{GECODE}} & \multicolumn{4}{c|}{\textbf{CHUFFED}} & \multicolumn{4}{c}{\textbf{CP-SAT}} \\
\cmidrule(lr){2-5}\cmidrule(lr){6-9}\cmidrule(lr){10-13}
  & bs & complete & noIMPL & noSB & bs & complete & noIMPL & noSB & bs & complete & noIMPL & noSB \\
\midrule
6 & \textbf{1} & \textbf{1} & \textbf{1} & \textbf{1} & \textbf{1} & \textbf{1} & \textbf{1} & \textbf{1} & \textbf{1} & \textbf{1} & \textbf{1} & \textbf{1} \\
8 & \textbf{1} & \textbf{1} & \textbf{1} & \textbf{1} & \textbf{1} & \textbf{1} & \textbf{1} & \textbf{1} & \textbf{1} & \textbf{1} & \textbf{1} & \textbf{1} \\
10 & \textbf{1} & \textbf{1} & \textbf{1} & \textbf{1} & \textbf{1} & \textbf{1} & \textbf{1} & \textbf{1} & \textbf{1} & \textbf{1} & \textbf{1} & \textbf{1} \\
12 & \textbf{1} & \textbf{1} & \textbf{1} & \textbf{1} & \textbf{1} & \textbf{1} & \textbf{1} & \textbf{1} & \textbf{1} & \textbf{1} & \textbf{1} & \textbf{1} \\
14 & \textbf{1} & \textbf{1} & \textbf{1} & \textbf{1} & N/A & \textbf{1} & \textbf{1} & N/A & N/A & N/A & N/A & N/A \\
16 & 15 & 3 & \textbf{1} & \textbf{1} & N/A & N/A & N/A & N/A & N/A & N/A & N/A & N/A \\
18 & N/A & N/A & N/A & N/A & N/A & N/A & N/A & N/A & N/A & N/A & N/A & N/A \\
20 & N/A & N/A & N/A & N/A & N/A & N/A & N/A & N/A & N/A & N/A & N/A & N/A \\
22 & N/A & N/A & N/A & N/A & N/A & N/A & N/A & N/A & N/A & N/A & N/A & N/A \\
\bottomrule
\end{tabular}%
}
\caption{Objective values using \textit{Sequential Custom Search Strategy}}
\end{table}

\begin{table}[htbp]
\centering
\small
\resizebox{\textwidth}{!}{%
\begin{tabular}{c|cccc|cccc|cccc}
\toprule
\textbf{n} & \multicolumn{4}{c|}{\textbf{GECODE}} & \multicolumn{4}{c|}{\textbf{CHUFFED}} & \multicolumn{4}{c}{\textbf{CP-SAT}} \\
\cmidrule(lr){2-5}\cmidrule(lr){6-9}\cmidrule(lr){10-13}
  & bs & complete & noIMPL & noSB & bs & complete & noIMPL & noSB & bs & complete & noIMPL & noSB \\
\midrule
6 & \textbf{1} & \textbf{1} & \textbf{1} & \textbf{1} & \textbf{1} & \textbf{1} & \textbf{1} & \textbf{1} & \textbf{1} & \textbf{1} & \textbf{1} & \textbf{1} \\
8 & \textbf{1} & \textbf{1} & \textbf{1} & \textbf{1} & \textbf{1} & \textbf{1} & \textbf{1} & \textbf{1} & \textbf{1} & \textbf{1} & \textbf{1} & \textbf{1} \\
10 & \textbf{1} & \textbf{1} & \textbf{1} & \textbf{1} & \textbf{1} & \textbf{1} & \textbf{1} & \textbf{1} & \textbf{1} & \textbf{1} & \textbf{1} & \textbf{1} \\
12 & \textbf{1} & \textbf{1} & \textbf{1} & \textbf{1} & \textbf{1} & \textbf{1} & \textbf{1} & \textbf{1} & \textbf{1} & \textbf{1} & \textbf{1} & \textbf{1} \\
14 & \textbf{1} & \textbf{1} & \textbf{1} & \textbf{1} & N/A & N/A & \textbf{1} & N/A & N/A & N/A & N/A & N/A \\
16 & \textbf{1} & \textbf{1} & \textbf{1} & \textbf{1} & N/A & N/A & 13 & N/A & N/A & N/A & N/A & N/A \\
18 & N/A & \textbf{1} & N/A & \textbf{1} & N/A & N/A & N/A & N/A & N/A & N/A & N/A & N/A \\
20 & N/A & N/A & N/A & N/A & N/A & N/A & N/A & N/A & N/A & N/A & N/A & N/A \\
22 & N/A & N/A & N/A & N/A & N/A & N/A & N/A & N/A & N/A & N/A & N/A & N/A \\
\bottomrule
\end{tabular}%
}
\caption{Objective values using \textit{Relax-and-Reconstruct (LNS) Strategy}}
\end{table}

\paragraph{Results Interpretation} The experimental results show that the \texttt{CP-SAT} solver, with its default search strategy, consistently delivers the best performance. Moreover, combining Sequential Search with the Relax-and-Reconstruct (LNS) strategy improves \texttt{Gecode} by solving larger instances and finding valid or optimal solutions where the default did not succeed.
